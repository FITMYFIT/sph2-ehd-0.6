% Created 2019-10-19 Sat 10:46
% Intended LaTeX compiler: pdflatex
\documentclass[presentation]{beamer}
\usepackage[utf8]{inputenc}
\usepackage[T1]{fontenc}
\usepackage{graphicx}
\usepackage{grffile}
\usepackage{longtable}
\usepackage{wrapfig}
\usepackage{rotating}
\usepackage[normalem]{ulem}
\usepackage{amsmath}
\usepackage{textcomp}
\usepackage{amssymb}
\usepackage{capt-of}
\usepackage{hyperref}
\usetheme{default}
\author{stardiviner}
\date{\today}
\title{}
\hypersetup{
 pdfauthor={stardiviner},
 pdftitle={},
 pdfkeywords={},
 pdfsubject={},
 pdfcreator={Emacs 26.3 (Org mode 9.2.6)}, 
 pdflang={English}}
\begin{document}

\begin{frame}{Outline}
\tableofcontents
\end{frame}

\#include "SPHEHD.h"


CSPHEHD::CSPHEHD()
\{
\}

CSPHEHD::\textasciitilde{}CSPHEHD()
\{
\}

void CSPHEHD::Solve(CRegion \& Region)
\{
  CSPHPt * PtiPtr,*PtjPtr;
  CBasePt * BasePtPtr;
  CKnl * KnlPtr;
  double Exi,Exj;
  double Eyi,Eyj;
  double epsiloni, epsilonj;//epsilon of part
  double kappai, kappaj;//kappa of part
  double norm;
  double normx,normy;
  double normx1,normy1;
  unsigned int IDi,IDj,ID2;

//1.update eepsilon and ekappa
//Lopez 2011 use an interpolation scheme to calculate transition value between the two fluid with VOF colour
//here we try an direct interpolation using SPH, just like the interpolation of colour
for (unsigned int i=0;i<Region.\textsubscript{PtPairList.size}();i++)
  \{
    // if(Region.\textsubscript{PtPairList}[i].\textsubscript{Type}!=enSphbndptpair\&\&Region.\textsubscript{PtPairList}[i].\textsubscript{Type}!=enSPHDumPtPair)
    \{
      PtiPtr=(CSPHPt*)Region.\textsubscript{PtPairList}[i].\textsubscript{PtiPtr};
      PtjPtr=(CSPHPt*)Region.\textsubscript{PtPairList}[i].\textsubscript{PtjPtr};
      KnlPtr=\&Region.\textsubscript{KnlList}[i];

epsiloni=Region.\textsubscript{PartList}[PtiPtr->\textsubscript{PID}-1].\textsubscript{eEpsilon};
epsilonj=Region.\textsubscript{PartList}[PtjPtr->\textsubscript{PID}-1].\textsubscript{eEpsilon};

kappai=Region.\textsubscript{PartList}[PtiPtr->\textsubscript{PID}-1].\textsubscript{eKappa};
kappaj=Region.\textsubscript{PartList}[PtjPtr->\textsubscript{PID}-1].\textsubscript{eKappa};

PtiPtr->\textsubscript{eEpsilon}+=PtjPtr->\textsubscript{mrho}*KnlPtr->\textsubscript{W}*epsilonj;

PtiPtr->\textsubscript{eKappa}+=PtjPtr->\textsubscript{mrho}*KnlPtr->\textsubscript{W}*kappaj;

if(PtiPtr!=PtjPtr\&\&PtjPtr->\textsubscript{Type}==enSPHPt)
  \{
    PtjPtr->\textsubscript{eEpsilon}+=PtiPtr->\textsubscript{mrho}*KnlPtr->\textsubscript{W}*epsiloni;

  PtjPtr->\textsubscript{eKappa}+=PtiPtr->\textsubscript{mrho}*KnlPtr->\textsubscript{W}*kappai;
\}

      if(PtjPtr->\textsubscript{Type}==enDumPt)
        \{
          PtjPtr->\textsubscript{eEpsilon}=epsilonj;
          PtjPtr->\textsubscript{eKappa}=kappaj;
        \}
    \}
  \}
//!!!!!!!!!!!!!!!!!!!!!!!!!!!!!!!!!!!!!!!!!!!!!!!!!!!!!!!!!!!!!!!!!!!!!!!!!!!!!!!!!!!!!!!!!!!!!!
\emph{/temp, specify epsilon and kappa from the part, without transient region between two fluid
/} for(unsigned int i=0;i!=Region.\textsubscript{PtList.size}();++i)
\emph{/ \{
/}   BasePtPtr=\&Region.\textsubscript{PtList}[i];
\emph{/   BasePtPtr->\textsubscript{eEpsilon}=Region.\textsubscript{PartList}[BasePtPtr->\textsubscript{PID}-1].\textsubscript{eEpsilon};
/} \}
//2.solve charge density conservation equ Lopez 2011 Equ(21)
//2.1 solve deRho
for (unsigned int i=0;i<Region.\textsubscript{PtPairList.size}();i++)
  \{
    if(Region.\textsubscript{PtPairList}[i].\textsubscript{Type}!=enSPHBndPtPair\&\&Region.\textsubscript{PtPairList}[i].\textsubscript{Type}!=enSPHDumPtPair)
      \{
        PtiPtr=(CSPHPt*)Region.\textsubscript{PtPairList}[i].\textsubscript{PtiPtr};
        PtjPtr=(CSPHPt*)Region.\textsubscript{PtPairList}[i].\textsubscript{PtjPtr};
        KnlPtr=\&Region.\textsubscript{KnlList}[i];

if(Region.\textsubscript{ControlSPH}.\textsubscript{RunMod}==1)//Explicit
  \{

PtiPtr->\textsubscript{deRho}-=PtjPtr->\textsubscript{mrho}*((PtiPtr->\textsubscript{eKappa}*PtjPtr->\textsubscript{eEx}+PtjPtr->\textsubscript{eKappa}*PtiPtr->\textsubscript{eEx})*KnlPtr->\textsubscript{Wx}
                               +(PtiPtr->\textsubscript{eKappa}*PtjPtr->\textsubscript{eEy}+PtjPtr->\textsubscript{eKappa}*PtiPtr->\textsubscript{eEy})*KnlPtr->\textsubscript{Wy});
PtiPtr->\textsubscript{deRho}-=PtjPtr->\textsubscript{mrho}*((PtiPtr->\textsubscript{eRho}*PtjPtr->\textsubscript{u}+PtjPtr->\textsubscript{eRho}*PtiPtr->\textsubscript{u})*KnlPtr->\textsubscript{Wx}
                               +(PtiPtr->\textsubscript{eRho}*PtjPtr->\textsubscript{v}+PtjPtr->\textsubscript{eRho}*PtiPtr->\textsubscript{v})*KnlPtr->\textsubscript{Wy});

if((PtiPtr!=PtjPtr)\&\&PtjPtr->\textsubscript{Type}==enSPHPt)
  \{
    PtjPtr->\textsubscript{deRho}+=PtiPtr->\textsubscript{mrho}*((PtjPtr->\textsubscript{eKappa}*PtiPtr->\textsubscript{eEx}+PtiPtr->\textsubscript{eKappa}*PtjPtr->\textsubscript{eEx})*KnlPtr->\textsubscript{Wx}
                                   +(PtjPtr->\textsubscript{eKappa}*PtiPtr->\textsubscript{eEy}+PtiPtr->\textsubscript{eKappa}*PtjPtr->\textsubscript{eEy})*KnlPtr->\textsubscript{Wy});
    PtjPtr->\textsubscript{deRho}+=PtiPtr->\textsubscript{mrho}*((PtjPtr->\textsubscript{eRho}*PtiPtr->\textsubscript{u}+PtiPtr->\textsubscript{eRho}*PtjPtr->\textsubscript{u})*KnlPtr->\textsubscript{Wx}
                                   +(PtjPtr->\textsubscript{eRho}*PtiPtr->\textsubscript{v}+PtiPtr->\textsubscript{eRho}*PtjPtr->\textsubscript{v})*KnlPtr->\textsubscript{Wy});

    \}
\}

      else//Implicit
        \{
          IDi=PtiPtr->\textsubscript{ID2};
        \}
    \}
\}

//2.2 update volume charge density eRho, rhoe n+1
double DeltaT;
DeltaT=Region.\textsubscript{ControlSPH}.\textsubscript{DeltaT};
for(unsigned int i=0; i<Region.\textsubscript{PtList.size}();i++)
  \{
    BasePtPtr=\&Region.\textsubscript{PtList}[i];
    if(BasePtPtr->\textsubscript{Type}==enSPHPt)
      \{
        BasePtPtr->\textsubscript{eRho}+=DeltaT*BasePtPtr->\textsubscript{deRho};
      \}
  \}

//3. calculate electric field E
//3.1 calculate φ, Lopez2011 Equ.(22),Poisson equ.
\emph{/3.1.1 define and initialize the matrix A and right hand side b, here in order to use lapack, we use array instead of vector
/}---------------------------------------------------------------------------------------------------------------------
unsigned int PtNum=Region.\textsubscript{CalList.size}();
//use laspack to solve linear equ.
QMatrix LasA;//laspack amtrix A
Vector Lasx,Lasb;//laspack vector, not std vector
size\textsubscript{t} Dim=PtNum;

char namea[]="A";
char namex[]="x";
char nameb[]="b";
Q\textsubscript{Constr}(\&LasA, namea, Dim , False , Rowws, Normal , True );
V\textsubscript{Constr}(\&Lasx, namex, Dim, Normal, True);
V\textsubscript{Constr}(\&Lasb, nameb, Dim, Normal, True);

//set the length of LasA, after q\textsubscript{setlen}, element will be set to 0
//for i, some neighbor will not involved in calculation (if neighbor j is dumpt or not in calculate range)
//so the length of row may larger than the element number, but it does not influence the result of Ax=b
for(unsigned int i=0;i!=Region.\textsubscript{CalList.size}();++i)
  \{
    BasePtPtr=\&Region.\textsubscript{PtList}[Region.\textsubscript{CalList}[i]];

Q\textsubscript{SetLen}(\&LasA, BasePtPtr->\textsubscript{ID2}, BasePtPtr->\textsubscript{NumNegbor});

    Q\textsubscript{SetEntry}(\&LasA, BasePtPtr->\textsubscript{ID2}, 0, BasePtPtr->\textsubscript{ID2}, 0.0);//set the first one of row as i-i element
  \}
//set all components of lasvector 0
V\textsubscript{SetAllCmp}(\&Lasb, 0.0);

\emph{/3.1.1 done
/}---------------------------------------------------------------------------------------------------------------------

\emph{/3.1.2 contruct and solove Ax=b
/}\[ - {\left( {{\rho _e}} \right)_a} = \sum\limits_b {\frac{{{m_b}}}{{{\rho _b}}}} ({\varepsilon _a} + {\varepsilon _b})({\phi _a} - {\phi _b})\frac{1}{{{r_{ab}}}}\frac{{{\partial _a}{W_{ab}}}}{{\partial {r_{ab}}}}\]
//the source term, b
for(unsigned int i=0;i!=Region.\textsubscript{CalList.size}();++i)
  \{
    BasePtPtr=\&Region.\textsubscript{PtList}[Region.\textsubscript{CalList}[i]];
    V\textsubscript{AddCmp}(\&Lasb, BasePtPtr->\textsubscript{ID2}, -BasePtPtr->\textsubscript{eRho});
  \}

//the matrix A
vector<unsigned int> icount;//element number of each row
icount.resize(PtNum,1);//there is one element now, i-i, the first one of each row

for(unsigned int i=0;i<Region.\textsubscript{PtPairList.size}();i++)
  \{
    if(Region.\textsubscript{PtPairList}[i].\textsubscript{Type}==enSPHPtPair||Region.\textsubscript{PtPairList}[i].\textsubscript{Type}==enSPHDumPtPair)
  \{
        PtiPtr=(CSPHPt*)Region.\textsubscript{PtPairList}[i].\textsubscript{PtiPtr};
        PtjPtr=(CSPHPt*)Region.\textsubscript{PtPairList}[i].\textsubscript{PtjPtr};
        KnlPtr=\&Region.\textsubscript{KnlList}[i];

//as (phi i-phi j ) is calculated,  so ptiptr=ptjptr makes no sense
if(PtiPtr!=PtjPtr)
  \{
    IDi=PtiPtr->\textsubscript{ID2};
    IDj=PtjPtr->\textsubscript{ID2};

norm=(PtiPtr->\textsubscript{eEpsilon}+PtjPtr->\textsubscript{eEpsilon})*KnlPtr->\textsubscript{Ww};
// norm=4*(PtiPtr->\textsubscript{eEpsilon}*PtjPtr->\textsubscript{eEpsilon})/(PtiPtr->\textsubscript{eEpsilon}+PtjPtr->\textsubscript{eEpsilon})*KnlPtr->\textsubscript{Ww};//Monaghan 2005 Equ(7.4)

Q\textsubscript{AddVal}(\&LasA, IDi, 0, PtjPtr->\textsubscript{mrho}*norm);//the first one in row is i i

if(PtjPtr->\textsubscript{Type}==enSPHPt)
  \{
    icount[IDi-1]++;

Q\textsubscript{SetEntry}(\&LasA, IDi, icount[IDi-1]-1, IDj, -PtjPtr->\textsubscript{mrho}*norm);

icount[IDj-1]++;

Q\textsubscript{AddVal}(\&LasA, IDj, 0, PtiPtr->\textsubscript{mrho}*norm);//the first one in row j-j
Q\textsubscript{SetEntry}(\&LasA, IDj, icount[IDj-1]-1, IDi, -PtiPtr->\textsubscript{mrho}*norm);

\}

            if(PtjPtr->\textsubscript{Type}==enDumPt)
              \{
                V\textsubscript{AddCmp}(\&Lasb, IDi, PtjPtr->\textsubscript{mrho}*norm*PtjPtr->\textsubscript{ePhi});
              \}
          \}
      \}
  \}
icount.clear();

SetRTCAccuracy(1e-8);
V\textsubscript{SetAllCmp}(\&Lasx, 0.0);
\emph{/ CGIter(\&LasA, \&Lasx, \&Lasb, 100, SSORPrecond,1.2);
/} CGIter(\&LasA, \&Lasx, \&Lasb, 20, SSORPrecond,1.2);
\emph{/ BiCGIter(\&LasA, \&Lasx, \&Lasb, 100, SSORPrecond,1.2);
/} BiCGIter(\&LasA, \&Lasx, \&Lasb, 20, SSORPrecond,1.2);
// BiCGSTABIter(\&LasA , \&Lasx, \&Lasb, 100, SSORPrecond,1.2);
BiCGSTABIter(\&LasA, \&Lasx, \&Lasb, 100, SSORPrecond,1.2);//the 4th parameter 100, is the maxiter steps
// outputV( Lasx,  "xxbicgs-20");
//output2(LasA, Lasx, Lasb, "xxbicgs-ax=b");

//set φ of particle from b
for(unsigned int i=0;i!=Region.\textsubscript{CalList.size}();++i)
  \{
    BasePtPtr=\&Region.\textsubscript{PtList}[Region.\textsubscript{CalList}[i]];

  BasePtPtr->\textsubscript{ePhi}=V\textsubscript{GetCmp}(\&Lasx, BasePtPtr->\textsubscript{ID2});
\}

Q\textsubscript{Destr}(\&LasA);
V\textsubscript{Destr}(\&Lasx);
V\textsubscript{Destr}(\&Lasb);

//3.1.2 done
\emph{/3.1 done
/}---------------------------------------------------------------------------------------------------------------------
//3.2 calculate E using CSPM, as the dirct scheme produce unacceptable errors
//3.2.1 cspm coefficient matrix
if(Region.\textsubscript{ControlSPH}.\textsubscript{CSPMIflag2}==0)
  \{
    \_CSPMEHD.GetCSPMGradCorctCoef(Region);//dummy particle involved
  \}

// //3.2.2 cspm source term
vector<double> bx;//CSPM修正的源项
vector<double> by;
bx.resize(Region.\textsubscript{CalList.size}(),0.0);
by.resize(Region.\textsubscript{CalList.size}(),0.0);
for(unsigned int i=0;i<Region.\textsubscript{PtPairList.size}();i++)
  \{
    if(Region.\textsubscript{PtPairList}[i].\textsubscript{Type}==enSPHPtPair// ||Region.\textsubscript{PtPairList}[i].\textsubscript{Type}==enSPHDumPtPair
       )
      \{
        PtiPtr=(CSPHPt*)Region.\textsubscript{PtPairList}[i].\textsubscript{PtiPtr};
        PtjPtr=(CSPHPt*)Region.\textsubscript{PtPairList}[i].\textsubscript{PtjPtr};
        KnlPtr=\&Region.\textsubscript{KnlList}[i];

IDi=PtiPtr->\textsubscript{ID2};
IDj=PtjPtr->\textsubscript{ID2};

if(PtiPtr!=PtjPtr)
  \{
    bx[IDi-1]+=PtjPtr->\textsubscript{mrho}*(PtiPtr->\textsubscript{ePhi}-PtjPtr->\textsubscript{ePhi})*KnlPtr->\textsubscript{Wx};
    by[IDi-1]+=PtjPtr->\textsubscript{mrho}*(PtiPtr->\textsubscript{ePhi}-PtjPtr->\textsubscript{ePhi})*KnlPtr->\textsubscript{Wy};

          //particle j
          if(PtjPtr->\textsubscript{Type}==enSPHPt)
            \{
              bx[IDj-1]+=PtiPtr->\textsubscript{mrho}*(PtiPtr->\textsubscript{ePhi}-PtjPtr->\textsubscript{ePhi})*KnlPtr->\textsubscript{Wx};
              by[IDj-1]+=PtiPtr->\textsubscript{mrho}*(PtiPtr->\textsubscript{ePhi}-PtjPtr->\textsubscript{ePhi})*KnlPtr->\textsubscript{Wy};
            \}
        \}
    \}
\}

//calculate CSPM corrected E=grad(phi)
for(unsigned int i=0;i!=Region.\textsubscript{CalList.size}();++i)
  \{
    BasePtPtr=\&Region.\textsubscript{PtList}[Region.\textsubscript{CalList}[i]];
    ID2=BasePtPtr->\textsubscript{ID2};

  BasePtPtr->\textsubscript{eEx}=-(BasePtPtr->\textsubscript{CSPMAxx}*bx[ID2-1]+BasePtPtr->\textsubscript{CSPMAyx}*by[ID2-1]);
  BasePtPtr->\textsubscript{eEy}=-(BasePtPtr->\textsubscript{CSPMAxy}*bx[ID2-1]+BasePtPtr->\textsubscript{CSPMAyy}*by[ID2-1]);
\}

bx.clear();
by.clear();


//3.3 calculate the accelerate caused by ehd force
//fomulation details are in Richard's STUDY DIRARY "2019年9月13日星期五" part 4 (similar formulation with Lopez 2011 Equ32)
\emph{/use CSPM
/}---------------------------------------------------------------------------------------------------------------------
//3.2 calculate electric force Fe using CSPM
//3.2.1 cspm coefficient matrix
if(Region.\textsubscript{ControlSPH}.\textsubscript{CSPMIflag2}==0)
  \{
    \_CSPMEHD.GetCSPMGradCorctCoef(Region);//dummy particle involved
  \}

//cspm source term
bx.resize(Region.\textsubscript{CalList.size}(),0.0);
by.resize(Region.\textsubscript{CalList.size}(),0.0);
for(unsigned int i=0;i<Region.\textsubscript{PtPairList.size}();i++)
  \{
    if(Region.\textsubscript{PtPairList}[i].\textsubscript{Type}==enSPHPtPair
       // ||Region.\textsubscript{PtPairList}[i].\textsubscript{Type}==enSPHDumPtPair
       )
      \{
        PtiPtr=(CSPHPt*)Region.\textsubscript{PtPairList}[i].\textsubscript{PtiPtr};
        PtjPtr=(CSPHPt*)Region.\textsubscript{PtPairList}[i].\textsubscript{PtjPtr};
        KnlPtr=\&Region.\textsubscript{KnlList}[i];

IDi=PtiPtr->\textsubscript{ID2};
IDj=PtjPtr->\textsubscript{ID2};

epsiloni=PtiPtr->\textsubscript{eEpsilon};
epsilonj=PtjPtr->\textsubscript{eEpsilon};

Exi=PtiPtr->\textsubscript{eEx};
Eyi=PtiPtr->\textsubscript{eEy};
Exj=PtjPtr->\textsubscript{eEx};
Eyj=PtjPtr->\textsubscript{eEy};

normx=epsiloni*(Exi*(Exi*KnlPtr->\textsubscript{Wx}+Eyi*KnlPtr->\textsubscript{Wy})-0.5*(Exi*Exi+Eyi*Eyi)*KnlPtr->\textsubscript{Wx})
  -epsilonj*(Exj*(Exj*KnlPtr->\textsubscript{Wx}+Eyj*KnlPtr->\textsubscript{Wy})-0.5*(Exj*Exj+Eyj*Eyj)*KnlPtr->\textsubscript{Wx});
normy=epsiloni*(Eyi*(Exi*KnlPtr->\textsubscript{Wx}+Eyi*KnlPtr->\textsubscript{Wy})-0.5*(Exi*Exi+Eyi*Eyi)*KnlPtr->\textsubscript{Wy})
  -epsilonj*(Eyj*(Exj*KnlPtr->\textsubscript{Wx}+Eyj*KnlPtr->\textsubscript{Wy})-0.5*(Exj*Exj+Eyj*Eyj)*KnlPtr->\textsubscript{Wy});

if(PtiPtr!=PtjPtr)
  \{
    bx[IDi-1]+=PtjPtr->\textsubscript{mrho}*normx;
    by[IDi-1]+=PtjPtr->\textsubscript{mrho}*normy;

          //particle j
          if(PtjPtr->\textsubscript{Type}==enSPHPt)
            \{
              bx[IDj-1]+=PtiPtr->\textsubscript{mrho}*normx;
              by[IDj-1]+=PtiPtr->\textsubscript{mrho}*normy;
            \}
        \}
    \}
\}

//calculate CSPM corrected electric force
for(unsigned int i=0;i!=Region.\textsubscript{CalList.size}();++i)
  \{
    BasePtPtr=\&Region.\textsubscript{PtList}[Region.\textsubscript{CalList}[i]];
    ID2=BasePtPtr->\textsubscript{ID2};

  BasePtPtr->\textsubscript{Fex}=BasePtPtr->\textsubscript{CSPMAxx}*bx[ID2-1]+BasePtPtr->\textsubscript{CSPMAyx}*by[ID2-1];
  BasePtPtr->\textsubscript{Fey}=BasePtPtr->\textsubscript{CSPMAxy}*bx[ID2-1]+BasePtPtr->\textsubscript{CSPMAyy}*by[ID2-1];
\}

bx.clear();
by.clear();
\emph{/ for(unsigned int i=0;i<Region.\textsubscript{PtPairList.size}();i++)
/}   \{
\emph{/     if(Region.\textsubscript{PtPairList}[i].\textsubscript{Type}==enSPHPtPair)
/}       \{
\emph{/         PtiPtr=(CSPHPt*)Region.\textsubscript{PtPairList}[i].\textsubscript{PtiPtr};
/}         PtjPtr=(CSPHPt*)Region.\textsubscript{PtPairList}[i].\textsubscript{PtjPtr};
//         KnlPtr=\&Region.\textsubscript{KnlList}[i];

\emph{/         epsiloni=PtiPtr->\textsubscript{eEpsilon};
/}         epsilonj=PtjPtr->\textsubscript{eEpsilon};

\emph{/         Exi=PtiPtr->\textsubscript{eEx};
/}         Eyi=PtiPtr->\textsubscript{eEy};
\emph{/         Exj=PtjPtr->\textsubscript{eEx};
/}         Eyj=PtjPtr->\textsubscript{eEy};

\emph{/         normx=-epsiloni*(Exi*(Exi*KnlPtr->\textsubscript{Wx}+Eyi*KnlPtr->\textsubscript{Wy})-0.5*(Exi*Exi+Eyi*Eyi)*KnlPtr->\textsubscript{Wx})
/}           +epsilonj*(Exj*(Exj*KnlPtr->\textsubscript{Wx}+Eyj*KnlPtr->\textsubscript{Wy})-0.5*(Exj*Exj+Eyj*Eyj)*KnlPtr->\textsubscript{Wx});
\emph{/         normy=-epsiloni*(Eyi*(Exi*KnlPtr->\textsubscript{Wx}+Eyi*KnlPtr->\textsubscript{Wy})-0.5*(Exi*Exi+Eyi*Eyi)*KnlPtr->\textsubscript{Wy})
/}           +epsilonj*(Eyj*(Exj*KnlPtr->\textsubscript{Wx}+Eyj*KnlPtr->\textsubscript{Wy})-0.5*(Exj*Exj+Eyj*Eyj)*KnlPtr->\textsubscript{Wy});

\emph{/         PtiPtr->\textsubscript{Fex}+=PtjPtr->\textsubscript{mrho}*normx;
/}         PtiPtr->\textsubscript{Fey}+=PtjPtr->\textsubscript{mrho}*normy;

\emph{/         if(PtiPtr!=PtjPtr\&\&PtjPtr->\textsubscript{Type}==enSPHPt)
/}           \{
\emph{/             PtjPtr->\textsubscript{Fex}+=PtiPtr->\textsubscript{mrho}*normx;
/}             PtjPtr->\textsubscript{Fey}+=PtiPtr->\textsubscript{mrho}*normy;
\emph{/           \}
/}       \}
//   \}


CKFile kfiletemp;
  kfiletemp.outTecplotEHDPLANNER(Region,1,"xxehd-case1-4");

  \emph{/ //two list of particles around the seperate line, for pressure jump
/} \emph{/4 layer each part, the list is corresponding to the particle assemble
/} unsigned int PtListUpper[6]=\{596,630,664,698,732,766\};
\emph{/ unsigned int PtListLower[6]=\{392,426,460,494,528,562\};
/} double plower=0.0;
\emph{/ double pupper=0.0;
/} for(unsigned int i=0;i!=6;++i)
\emph{/   \{
/}     pupper+=0.5*pow(Region.\textsubscript{PtList}[PtListUpper[i]-1].\textsubscript{eEy,2});
\emph{/     plower+=0.5*pow(Region.\textsubscript{PtList}[PtListLower[i]-1].\textsubscript{eEy,2});
/}   \}

  // cout<<"Pressure Jump is :"<<(plower-pupper)/pupper<<endl;
\}




void CSPHEHD::Solve2(CRegion \& Region)
\{
  CSPHPt * PtiPtr,*PtjPtr;
  CBasePt * BasePtPtr;
  CKnl * KnlPtr;
  double eEpsiloni, eEpsilonj;//epsilon of part
  double eKappai, eKappaj;//kappa of part
  double norm;
  double normx,normy;
  double normx1,normy1;
  unsigned int IDi,IDj;  

//1.update eepsilon and ekappa
//Lopez 2011 use an interpolation scheme to calculate transition value between the two fluid with VOF colour
//here we try an direct interpolation using SPH, just like the interpolation of colour
for (unsigned int i=0;i<Region.\textsubscript{PtPairList.size}();i++)
  \{
    // if(Region.\textsubscript{PtPairList}[i].\textsubscript{Type}!=enSphbndptpair\&\&Region.\textsubscript{PtPairList}[i].\textsubscript{Type}!=enSPHDumPtPair)
    \{
      PtiPtr=(CSPHPt*)Region.\textsubscript{PtPairList}[i].\textsubscript{PtiPtr};
      PtjPtr=(CSPHPt*)Region.\textsubscript{PtPairList}[i].\textsubscript{PtjPtr};
      KnlPtr=\&Region.\textsubscript{KnlList}[i];

eEpsiloni=Region.\textsubscript{PartList}[PtiPtr->\textsubscript{PID}-1].\textsubscript{eEpsilon};
eEpsilonj=Region.\textsubscript{PartList}[PtjPtr->\textsubscript{PID}-1].\textsubscript{eEpsilon};

eKappai=Region.\textsubscript{PartList}[PtiPtr->\textsubscript{PID}-1].\textsubscript{eKappa};
eKappaj=Region.\textsubscript{PartList}[PtjPtr->\textsubscript{PID}-1].\textsubscript{eKappa};

PtiPtr->\textsubscript{eEpsilon}+=PtjPtr->\textsubscript{mrho}*KnlPtr->\textsubscript{W}*eEpsilonj;

PtiPtr->\textsubscript{eKappa}+=PtjPtr->\textsubscript{mrho}*KnlPtr->\textsubscript{W}*eKappaj;

if(PtiPtr!=PtjPtr)
  \{
    PtjPtr->\textsubscript{eEpsilon}+=PtiPtr->\textsubscript{mrho}*KnlPtr->\textsubscript{W}*eEpsiloni;

        PtjPtr->\textsubscript{eKappa}+=PtiPtr->\textsubscript{mrho}*KnlPtr->\textsubscript{W}*eKappai;
      \}
  \}
\}

//2.calulate phi n+1, construct A(phi n+1)=b, nabla((K*deltaT+epsilon)*nabla(phi n+1))=-rhoe n
double DeltaT;
DeltaT=Region.\textsubscript{ControlSPH}.\textsubscript{DeltaT};
unsigned int PtNum=Region.\textsubscript{CalList.size}();//the number used to initializae the size of linear equation
double ** A;
double * b;

//allocate memory dynamically
//defaultly, elements in A and b will be initialized as 0
b=new double[PtNum];
A = (double**) new double* [PtNum];
for (unsigned int i = 0; i != PtNum; ++i)
  \{
    A[i] = new double[PtNum];
  \}

for(unsigned int i=0;i!=PtNum;++i)
  \{
    b[i]=0.0;

  for(unsigned int j=0;j!=PtNum;++j)
    \{
      A[i][j]=0.0;
    \}
\}

for (unsigned int i=0;i<Region.\textsubscript{PtPairList.size}();i++)                
  \{                                                                   
    if(Region.\textsubscript{PtPairList}[i].\textsubscript{Type}==enSPHPtPair||Region.\textsubscript{PtPairList}[i].\textsubscript{Type}==enSPHDumPtPair)
      \{
        PtiPtr=(CSPHPt*)Region.\textsubscript{PtPairList}[i].\textsubscript{PtiPtr};
        PtjPtr=(CSPHPt*)Region.\textsubscript{PtPairList}[i].\textsubscript{PtjPtr};
        KnlPtr=\&Region.\textsubscript{KnlList}[i];

if(Region.\textsubscript{ControlSPH}.\textsubscript{RunMod}==1)//Explicit
  \{
    IDi=PtiPtr->\textsubscript{ID2};
    IDj=PtjPtr->\textsubscript{ID2};

//K*deltaT+epsilon
norm=KnlPtr->\textsubscript{Ww}*(PtiPtr->\textsubscript{eKappa}*DeltaT+PtiPtr->\textsubscript{eEpsilon}+PtjPtr->\textsubscript{eKappa}*DeltaT+PtjPtr->\textsubscript{eEpsilon});    

A[IDi-1][IDi-1]+=PtjPtr->\textsubscript{mrho}*norm;

if(PtjPtr->\textsubscript{Type}==enSPHPt)
  \{
    A[IDj-1][IDi-1]=-PtjPtr->\textsubscript{mrho}*norm;

      if(PtiPtr!=PtjPtr)
        \{
          A[IDj-1][IDj-1]+=PtiPtr->\textsubscript{mrho}*norm;
          A[IDi-1][IDj-1] =PtiPtr->\textsubscript{mrho}*norm;
        \}
    \}
  if(PtjPtr->\textsubscript{Type}==enDumPt)
    \{
      b[IDi-1]+=PtjPtr->\textsubscript{mrho}*norm*PtjPtr->\textsubscript{ePhi};
    \}
\}


        else//Implicit
          \{
            IDi=PtiPtr->\textsubscript{ID2};
          \}
      \}
  \}
for(unsigned int i=0;i!=Region.\textsubscript{CalList.size}();++i)
  \{
    BasePtPtr=\&Region.\textsubscript{PtList}[Region.\textsubscript{CalList}[i]-1];

  b[BasePtPtr->\textsubscript{ID2}-1]-=BasePtPtr->\textsubscript{eRho};
\}

int n=PtNum;
int nrhs=1;//number of rows of b, can more than 1
int lda=PtNum;//leading dimension of A, lda>=max(1,n)
int ldb=PtNum;//leading dimension of b, ldb>=max(1,n)
int *ipiv;//pivot row exchange records, line i is exchanged to line ipiv[i] (not understand yet)
int info;

ipiv=new int[PtNum];

\emph{/calculate Ax=b using dgesv\_ in LAPACK, after that ,b will be x, electric potential φ
/} dgesv\textsubscript{(\&n, \&nrhs, \&A[0][0], \&lda, ipiv, \&b[0], \&ldb, \&info)};

if(info!=0)//dgesv\_ failed, actually, a return is needed to stop the whole program
  cout<<"EHD linear equ Ax=b calculation fialed!"<<endl;

//set φ n+1 of particle from b
for(unsigned int i=0;i!=Region.\textsubscript{CalList.size}();++i)
  \{
    BasePtPtr=\&Region.\textsubscript{PtList}[Region.\textsubscript{CalList}[i]-1];

  BasePtPtr->\textsubscript{ePhi}=b[BasePtPtr->\textsubscript{ID2}-1];
\}

//


//free the memory
delete[] A;
delete[] b;
delete[] ipiv;

//calculate E n+1=grad(phi n+1)
for (unsigned int i=0;i<Region.\textsubscript{PtPairList.size}();i++)
  \{
    if(Region.\textsubscript{PtPairList}[i].\textsubscript{Type}==enSPHPtPair||Region.\textsubscript{PtPairList}[i].\textsubscript{Type}==enSPHDumPtPair)
     \{
       PtiPtr=(CSPHPt*)Region.\textsubscript{PtPairList}[i].\textsubscript{PtiPtr};
       PtjPtr=(CSPHPt*)Region.\textsubscript{PtPairList}[i].\textsubscript{PtjPtr};
       KnlPtr=\&Region.\textsubscript{KnlList}[i];

PtiPtr->\textsubscript{eEpsilon}+=PtjPtr->\textsubscript{mrho}*KnlPtr->\textsubscript{W}*eEpsilonj;

PtiPtr->\textsubscript{eKappa}+=PtjPtr->\textsubscript{mrho}*KnlPtr->\textsubscript{W}*eKappaj;

if(PtiPtr!=PtjPtr)
    \{
      PtjPtr->\textsubscript{eEpsilon}+=PtiPtr->\textsubscript{mrho}*KnlPtr->\textsubscript{W}*eEpsiloni;

           PtjPtr->\textsubscript{eKappa}+=PtiPtr->\textsubscript{mrho}*KnlPtr->\textsubscript{W}*eKappai;
         \}
    \}
\}

\}




void CSPHEHD::output(double ** a, double * b, int n)
\{
  ofstream outfile;
  ostringstream outfilename;
  outfilename<<"Matrix-ehd"<<".dat";
  outfile.open(outfilename.str().data(),ios::out);

if (!outfile)
  \{
    cerr<<"Out put file could not be open."<<endl<<"tecplot.dat"<<endl;
    return;
  \}

for (size\textsubscript{t} i=0;i<n;i++)
  \{
    for(size\textsubscript{t} j=0;j!=n;++j)
      \{
        outfile
          <<setiosflags(ios\textsubscript{base}::scientific)
          <<setprecision(16)
          <<setw(26)<<a[j][i]<<" ";
      \}

outfile
  <<setiosflags(ios\textsubscript{base}::scientific)
  <<setprecision(16)
  <<setw(26)<<b[i]<<" "
  <<endl;

\}


cout<<"Output file has been written."<<endl;

  outfile.close();
\}


void CSPHEHD::output2(QMatrix \& a, Vector \& x, Vector \& b ,string outputname)
\{

ofstream outfile;
ostringstream outfilename;
outfilename<<outputname<<".dat";
outfile.open(outfilename.str().data(),ios::out);

if (!outfile)
  \{
    cerr<<"Out put file could not be open."<<endl<<"tecplot.dat"<<endl;
    return;
  \}

size\textsubscript{t} n=Q\textsubscript{GetDim}(\&a);
for (size\textsubscript{t} i=0;i<n;i++)
  \{
    for(size\textsubscript{t} j=0;j!=n;++j)
      \{
        outfile
          <<setiosflags(ios\textsubscript{base}::scientific)
          <<setprecision(16)
          <<setw(26)<<Q\textsubscript{GetEl}(\&a, i+1, j+1)<<" ";
      \}

outfile
  <<setiosflags(ios\textsubscript{base}::scientific)
  <<setprecision(16)
  <<setw(26)<<V\textsubscript{GetCmp}(\&x, i+1)<<" "
  <<setw(26)<<V\textsubscript{GetCmp}(\&b, i+1)<<" "
  <<endl;

\}


cout<<"Output2 file has been written."<<endl;

   outfile.close();
\}


void CSPHEHD::outputV(Vector \& x ,string outputname)
\{

ofstream outfile;
ostringstream outfilename;
outfilename<<outputname<<".dat";
outfile.open(outfilename.str().data(),ios::out);

if (!outfile)
  \{
    cerr<<"Out put file could not be open."<<endl<<"tecplot.dat"<<endl;
    return;
  \}

size\textsubscript{t} n=V\textsubscript{GetDim}(\&x);
for (size\textsubscript{t} i=0;i<n;i++)
  \{
    outfile
      <<setiosflags(ios\textsubscript{base}::scientific)
      <<setprecision(16)
      <<setw(26)<<V\textsubscript{GetCmp}(\&x, i+1)<<" "
      <<endl;
  \}


cout<<outputname<<" file has been written."<<endl;

  outfile.close();
\}
\end{document}
